\documentclass[12pt]{article}

%---------------- Packages ----------------
\usepackage[a4paper,margin=1in]{geometry}
\usepackage{hyperref}
\usepackage{array}
\usepackage{longtable}

\title{Project Mirage\\[0.5em]
\large Formal Objective Breakdown, System Goals, and Access Instructions}
\author{Group {11}}
\date{\today}

\begin{document}

\maketitle
\thispagestyle{empty}
\newpage

\tableofcontents
\newpage

%==================================================
\section{Formal Objective Breakdown}
%==================================================

Project Mirage aims to deliver \textbf{Anya}, a virtual assistant integrated into Microsoft Teams for the 
Santa Barbara Public Defender's Office (SBPD). The system centralizes complex and frequently used 
workflows such as IT support ticketing, administrative requests, file retrieval, and staff directory lookups.
The objective is to reduce friction, eliminate repetitive manual tasks, and unify access to multiple existing
SBPD tools through a single conversational interface.

The formal objectives are as follows:

\begin{enumerate}
    \item \textbf{Provide a unified point of access}: Ensure that users no longer need to navigate multiple 
          applications (PDHelpDesk, PDGo, SharePoint) individually.
    \item \textbf{Streamline ticket and request creation}: Reduce time spent on routine IT and 
          administrative tasks.
    \item \textbf{Improve information discoverability}: Allow users to search for files and staff contacts 
          through natural language queries inside Teams.
    \item \textbf{Maintain compatibility with existing systems}: Integrate seamlessly with SBPD’s current 
          Microsoft 365 ecosystem without requiring new infrastructure.
    \item \textbf{Enhance accuracy and consistency}: Minimize user error through guided prompts, 
          structured data collection, and automation.
    \item \textbf{Support future extensibility}: Provide a foundation for additional workflows and 
          enhancements beyond Snapshot~1.
\end{enumerate}

%==================================================
\section{Goals and Rationale (Why the System Is Needed)}
%==================================================

SBPD staff currently rely on several independent tools for daily operations. These tools are functional but 
disconnected, requiring users to remember which system handles which task and how to navigate its 
interface. This fragmentation increases cognitive load, slows down workflows, and contributes to 
inconsistency in record creation.

Project Mirage addresses these issues directly. The system is needed for the following reasons:

\subsection{Operational Efficiency}
Staff frequently submit IT tickets, administrative requests, and file searches. Consolidating these actions 
into a single conversational interface measurably reduces the time required to complete essential tasks.

\subsection{Ease of Use}
Many users are already familiar with Microsoft Teams. Embedding Anya within Teams removes the need 
for training on multiple applications and makes the assistant available in the same workspace used for 
daily communication.

\subsection{Reduction of Errors}
By structuring workflows through automated prompts, the system prevents incomplete or incorrectly 
formatted submissions, improving data quality and decreasing the need for follow-up corrections.

\subsection{Improved Information Access}
Legal staff often need rapid access to files, templates, directories, and policy documents. Natural 
language file search reduces the time spent digging through SharePoint hierarchies.

\subsection{Centralization of Support}
Users are no longer required to determine which system contains the function they need. Mirage 
automatically routes requests to the correct back-end service.

\subsection{Alignment With SBPD Digital Modernization Efforts}
Mirage demonstrates how low-code and AI-based tools can enhance workflow automation without 
requiring extensive redevelopment of existing infrastructure.

%==================================================
\section{How to Download, Access, or Use the System}
%==================================================

Anya is deployed fully within the Microsoft 365 environment and does not require manual installation.
Access is controlled through the SBPD Teams tenant and associated security permissions.

\subsection{Access Requirements}
\begin{itemize}
    \item An active SBPD Microsoft 365 account.
    \item Access to Microsoft Teams (desktop or web).
    \item Inclusion within the Teams environment where Anya is deployed.
\end{itemize}

\subsection{Steps to Access Anya in Microsoft Teams}
\begin{enumerate}
    \item Open \textbf{Microsoft Teams}.
    \item Navigate to the \textbf{Team} or \textbf{Channel} where Anya is installed.  
          For example: \textbf{SBPD -- Mirage Testing Channel}.
    \item In the chat input box, type a message such as: \\
          \textit{``Hello Anya''} or \textit{``I need help''}.
    \item Anya will respond with a greeting and optional suggestions for available commands.
    \item Users may begin issuing natural language instructions such as:
    \begin{itemize}
        \item \textit{``Create an IT ticket.''}
        \item \textit{``I want to request time off next Friday.''}
        \item \textit{``Find the HR manual.''}
        \item \textit{``Look up contact information for John Smith.''}
    \end{itemize}
\end{enumerate}

\subsection{No Local Installation Required}
Project Mirage does not include a downloadable application. All components---including Copilot logic, 
Power Automate flows, and PowerApps integrations---are hosted within the Microsoft 365 cloud 
platform. Users automatically receive updates as new features are deployed.

\subsection{Developer / Administrator Access}
For teams maintaining or extending Mirage:

\begin{itemize}
    \item Logic flows are accessed through \textbf{Power Automate}.
    \item Conversation design is available in \textbf{Copilot Studio}.
    \item Back-end data is managed through \textbf{PowerApps}, \textbf{SharePoint lists}, or 
          \textbf{Smartsheet}.
\end{itemize}

Proper permissions are required to modify any component.

%==================================================
\section{System Benefits Summary}
%==================================================

Project Mirage provides measurable improvements to SBPD operations:

\begin{itemize}
    \item Faster IT and administrative request handling.
    \item Fewer user errors in ticket and request submissions.
    \item Simplified access to organizational knowledge and file repositories.
    \item Unified workflow management through a single assistant.
    \item Immediate compatibility with Microsoft 365 without new hardware.
    \item Reduced training overhead for new staff.
\end{itemize}

These benefits support SBPD’s mission by allowing staff to focus on legal work rather than navigating 
multiple administrative systems.

%==================================================
\section{References}
%==================================================

\begin{itemize}
    \item Microsoft Teams documentation: \url{https://learn.microsoft.com/microsoftteams/}
    \item Microsoft PowerApps documentation: \url{https://learn.microsoft.com/power-apps/}
    \item Microsoft Copilot Studio documentation: \url{https://learn.microsoft.com/microsoft-copilot-studio/}
    \item Project Mirage SRS and SDD reference documents.
\end{itemize}

\end{document}
